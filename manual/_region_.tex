\message{ !name(skycam_manual.tex)}\documentclass[a4paper,12pt,leqno,notitlepage]{article}
\usepackage{cmap}
\usepackage[T2A]{fontenc}
\usepackage[utf8]{inputenc}
\usepackage[english,russian]{babel}
\usepackage{amssymb,amsfonts,amsmath,mathtext,textcomp}
\usepackage[unicode,bookmarksopen=true,pdfborder=0,pdfstartview=FitH]{hyperref}
\usepackage{graphicx}
\usepackage{wrapfig}
\usepackage{longtable}
%\usepackage{grffile}
\usepackage{epsfig}
%\usepackage[pdftex]{graphicx}
\usepackage{epstopdf}
\usepackage{verbatim}    %comments like \begin{comment}..... \end{comment}
\frenchspacing
\righthyphenmin=2
\sloppy


\usepackage[left=20mm,right=15mm,top=20mm,bottom=20mm,bindingoffset=0cm]{geometry}
\usepackage{setspace}
%\textwidth=15.3cm
%\textheight=25cm
%\oddsidemargin=1cm
%\evensidemargin=1cm
%\renewcommand{\captionlabeldelim}{.} %точка вместо двоеточия в номере таблиц и рисунков
%\headheight=0.1cm
%\topmargin=0.1cm
%\headsep=0.1cm
\begin{document}

\message{ !name(skycam_manual.tex) !offset(-3) }


\begin{center}
  {\Huge Инструкция к камере на телескопе.
}
\end{center}

\section{Введение}
Камера установлена на телескопе на одной направляющей с противовесами и показывает область неба
диаметром примерно 3.5\textdegree вокруг поля зрения телескопа. Изображение с камеры выводится на
экран ноутбука, установленного в наблюдательской. Основная идея состоит в том, чтобы заменить этой
камерой искатель телескопа и выполнять ``one star align'' из наблюдательской, выставляя телескоп
по яркой звезде, ориентируясь по изображению на экране. Кроме того, камера позволяет в течение
ночи контролировать небольшой кусочек неба на предмет облаков и тому подобного.

То есть общий план инициализации телескопа меняется следующим образом: после выставления телескопа
в полюс и включения питания, наблюдатель спускается вниз и совершает все оставшиеся действия из
наблюдательской. Питание Клавиши скоро будет сделано через провод и поднять ее наверх будет нельзя.

\section{Включение камеры}
При снятии чехла с телескопа помните, что на нем теперь висит еще и камера. Она установлена
довольно крепко, но излишних усилий лучше не прилагать.

Основные этапы включения камеры следующие.
\begin{itemize}
\item С объектива камеры нужно снять крышку.
\item В наблюдательской нужно включить в розетку блок питания камеры. Он должен лежать на столе
около компьютера, там же, где лежит вилка противоросника.
\item После этого надо включить ноутбук. На передней панели слева расположен серебристый рычажок, его нужно
потянуть влево. После загрузки ноутбука, трансляция с камеры запустится автоматически.
\end{itemize}

\section{Описание программы}
На большей части экрана ноутбука показывается изображение с камеры. Красным прямоугольником
отмечено расположение поля телескопа внутри поля камеры. Размер прямоугольника приблизительно
соответствует размеру поля зрения телескопа. Для выставления телескопа по звезде при помощи 
Клавиши или ``The Sky'' привести звезду в красный прямоугольник, а затем уже ловить ее в
``CCDOps'' (аналогично тому, как это делалось с искателем).

На данный момент программа управляется следующими кнопками:

\textbf{Стрелки вверх, вниз, влево, вправо}: позволяют двигать красный прямоугольник поля
зрения телескопа. Это может понадобиться на тот случай, если наводка камеры собьется и нужно
будет привести указатель поля зрения телескопа к реальному положению дел (все как с искателем).
Новое положение прямоугольника автоматически сохраняется и будет использовано при следующих
запусках программы до тех пор пока кто-то снова не передвинет его.

\textbf{Кнопки < и > (они же кнопки ``б'' и ``ю'')}: изменяют количество кадров, которые
программа суммирует прежде, чем отобразить изображение. По умолчанию программа суммирует
по 4 кадра. Нажатие кнопки ``<'' уменьшает количество кадров вдвое, нажатие кнопки ``>''
увеличивает вдвое. Текущее значение количества суммируемых кадров показано справа от изображения.
Если Вам кажется, что изображение шумновато, то попробуйте увеличить число кадров, значения
в районе 16 -- 32 обычно достаточно. Учтите, что с ростом числа суммируемых кадров увеличивается и время
между ними.

\textbf{Кнопка S}: нужна для принудительной подгонки темнового кадра. Специфика камеры
(отсутствие механического затвора и стабилизирующего температуру холодильника) немного усложняет
учет дарков. В двух словах дело обстоит следующим образом: программа содержит в памяти некоторый
средний дарк и в течение ночи пытается подогнать его под реальный (и недоступный нам) дарк, опираясь при этом
на то ``хорошо'' или ``плохо'' выглядит кадр после учета дарка. Программа делает это автоматически
раз в минуту, что, в принципе, должно хватать. Но если все-таки наблюдателю кажется, что дарк вычтен
плохо (явно видны горячие пикселы, либо, наоборот, яркие точки), то можно нажать кнопку \textbf{S}
и принудительно пересчитать дарк. (Если еще короче, то в большинстве случаев эта кнопка не нужна).

\textbf{Кнопка D}: нужна для записи нового опорного среднего дарка. Хотя алгоритм подгонки
обеспечивает приемлемую точность учета дарка, сам дарк может медленно меняться со временем.
Чтобы записать новый дарк нужно плотно закрыть камеру крышкой, установить большое число кадров
суммирования (хотя бы 32) и нажать кнопку D. После того как кадр на экране пару раз сменится,
можно считать, что новый дарк сохранен в память компьютера. Обращаю внимание, что речь идет
скорее о длительном 

\end{document}
\message{ !name(skycam_manual.tex) !offset(-106) }
