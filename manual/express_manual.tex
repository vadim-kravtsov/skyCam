\documentclass[a4paper,12pt,leqno,notitlepage]{article}
\usepackage{cmap}
\usepackage[T2A]{fontenc}
\usepackage[utf8]{inputenc}
\usepackage[english,russian]{babel}
\usepackage{amssymb,amsfonts,amsmath,mathtext,textcomp}
\usepackage[unicode,bookmarksopen=true,pdfborder=0,pdfstartview=FitH]{hyperref}
\usepackage{graphicx}
\usepackage{wrapfig}
\usepackage{longtable}
%\usepackage{grffile}
\usepackage{epsfig}
%\usepackage[pdftex]{graphicx}
\usepackage{epstopdf}
\usepackage{verbatim}    %comments like \begin{comment}..... \end{comment}
\frenchspacing
\righthyphenmin=2
\sloppy


\usepackage[left=20mm,right=15mm,top=20mm,bottom=20mm,bindingoffset=0cm]{geometry}
\usepackage{setspace}
%\textwidth=15.3cm
%\textheight=25cm
%\oddsidemargin=1cm
%\evensidemargin=1cm
%\renewcommand{\captionlabeldelim}{.} %точка вместо двоеточия в номере таблиц и рисунков
%\headheight=0.1cm
%\topmargin=0.1cm
%\headsep=0.1cm
\begin{document}

\begin{center}
  {\Huge \bf Экспресс-инструкция к погодной камере.
}
\end{center}

\section{{\huge Включение}}
\Large
\begin{itemize}
\item Снять крышку с камеры
\item Воткнуть камеру в розетку (вилка с надписью SkyCam)
\item Включить ноутбук (рычажок ``ON/OFF''). Трансляция запустится автоматически
(после загрузки ноутбука).
\end{itemize}

\section{{\huge Выключение}}
\Large

\begin{itemize}
\item Выйти из программы (кнопка ESC на ноутбуке)
\item Выключить ноутбук
\item Выдернуть камеру из розетки
\item Закрыть камеру крышкой
\item Крышку ноутбука не закрывать, ноутбук из розетки не выключать.
\end{itemize}
\end{document}